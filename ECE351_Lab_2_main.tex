%%%%%%%%%%%%%%%%%%%%%%%%%%%%%
% Carlos Santos             %    
% ECE 351-51                %
% Lab 2                     %
% 2/9/2020                 %
%                           %
%%%%%%%%%%%%%%%%%%%%%%%%%%%%%

\documentclass[12pt]{article}

% Language and font encoding
\usepackage[english]{babel}
\usepackage[utf8x]{inputenc}
\usepackage[T1]{fontenc}
\usepackage{graphicx}
\usepackage{amsmath}
\usepackage{caption}
\usepackage{float}
\usepackage{caption}
\usepackage{subcaption}
\usepackage{rotating}
\usepackage{setspace}


% Sets page size and margins
\usepackage[a4paper,top=3cm,bottom=2cm,left=3cm,right=3cm,marginparwidth=1.75cm]{geometry}

% Useful packages
\usepackage[colorinlistoftodos]{todonotes}
\usepackage[colorlinks=true, allcolors=blue]{hyperref}
\usepackage{listings}
\usepackage{gensymb}
\usepackage{graphicx} %package to manage images
\usepackage{listings}
\usepackage{color}

\definecolor{dkgreen}{rgb}{0,0.6,0}
\definecolor{gray}{rgb}{0.5,0.5,0.5}
\definecolor{mauve}{rgb}{0.58,0,0.82}

\lstset{frame=tb,
  language=Python,
  aboveskip=3mm,
  belowskip=3mm,
  showstringspaces=false,
  columns=flexible,
  basicstyle={\small\ttfamily},
  numbers=none,
  numberstyle=\tiny\color{gray},
  keywordstyle=\color{blue},
  commentstyle=\color{dkgreen},
  stringstyle=\color{mauve},
  breaklines=true,
  breakatwhitespace=true,
  tabsize=3
}

%Line Spacing
\setstretch{1.5}


%Info for Title Page
\title{ECE 351 Lab 2 Report \\ Section 52}
\date{February 11, 2020}
\author{Carlos Santos}
\begin{document}

%Make a Title Page
\vspace{\fill}
\maketitle
\vspace{\fill}
\clearpage

%Introduction
\section*{Introduction}

This labs purpose is to introduce Python functions. More specifically how to define and use them. We will use different signal methods like differentiation, time shifting, time scaling, and signal addition.

%Equations
\section*{Equations}

\begin{equation}
    y(t) = r(t) - r(t-3) + 5*u(t-3) - 2*u(t-6) - 2*r(t-6)
\end{equation}

%Methodology
\section*{Methodology}

The key to this lab is to break it up into smaller pieces. First we begin by testing out the sample code to make sure all the proper libraries are installed. Sample code will be listed below: \\

\begin{lstlisting}
// Sample_code.py
1 import numpy as np
2 import matplotlib . pyplot as plt
3
4 plt . rcParams . update ({ ’fontsize ’: 14}) # Set font size in plots
5
6 steps = 1e -2 # Define step size
7 t = np . arange (0 , 5 + steps , steps ) # Add a step size to make sure the
8 # plot includes 5.0. Since np. arange () only
9 # goes up to , but doesn ’t include the
10 # value of the second argument
11 print (’Number of elements : len(t) = ’, len( t ) , ’\ nFirst Element : t[0] = ’, t [0] , ’
\ nLast Element : t[len(t) - 1] = ’, t [len( t ) - 1])
12 # Notice the array might be a different size than expected since Python starts
13 # at 0. Then we will use our knowledge of indexing to have Python print the
14 # first and last index of the array . Notice the array goes from 0 to len () - 1
15
16 # --- User - Defined Function ---
17
18 # Create output y(t) using a for loop and if/ else statements
19 def example1 ( t ) : # The only variable sent to the function is t
20 y = np . zeros ( t . shape ) # initialze y(t) as an array of zeros
21
22 for i in range (len ( t ) ) : # run the loop once for each index of t
23 if i < ( len( t ) + 1) /3:
24 y [ i ] = t [ i ]**2
25 else :
26 y [ i ] = np . sin (5* t [ i ]) + 2
27 return y # send back the output stored in an array
28
29 y = example1 ( t ) # call the function we just created
30
31 plt . figure ( figsize = (10 , 7) )
32 plt . subplot (2 , 1 , 1)
33 plt . plot (t , y )
34 plt . grid ()
35 plt . ylabel (’y(t) with Good Resolution ’)
36 plt . title (’Background - Illustration of for Loops and if/ else Statements ’)
37
38 t = np . arange (0 , 5 + 0.25 , 0.25) # redefine t with poor resolution
39 y = example1 ( t )
40
41 plt . subplot (2 , 1 , 2)
42 plt . plot (t , y )
43 plt . grid ()
44 plt . ylabel (’y(t) with Poor Resolution ’)
45 plt . xlabel (’t’)
46 plt . show ()
\end{lstlisting}

From the above code the important pieces are as follows:

\begin{enumerate}
    \item Make sure to include lines 1 and 2. Those are important libraries to be able to graph stuff.
    \item The key to defining variables is: namegoeshere = valueofnamegoeshere. Python is smart enough to figure out the types.
    \item " print("abcdefg") " is your friend! 
    \item Loop example can be found on line 22.
\end{enumerate}


If the above code works we can begin to do the rest of this lab. We create functions to plot step, ramp, and derivative functions. They can then be combined to ultimately graph our equation 1.

%Results
\section*{Results}


\begin{figure}[H]
\caption{Section 3}
\centering
\includegraphics[width=.8\textwidth]{output_from_example_code.png}
\end{figure}

\begin{figure}[H]
\caption{Section 4}
\centering
\includegraphics[width=.8\textwidth]{cosine_funtion.png}
\end{figure}

\begin{figure}[H]
\caption{Section 5}
\centering
\includegraphics[width=.8\textwidth]{step_funtion.png}
\end{figure}


\begin{figure}[H]
\caption{Section 5}
\centering
\includegraphics[width=.8\textwidth]{ramp_funtion.png}
\end{figure}

\begin{figure}[H]
\caption{Section 5}
\centering
\includegraphics[width=.8\textwidth]{plotted_funtion.png}
\end{figure}


\begin{figure}[H]
\caption{Section 6}
\centering
\includegraphics[width=.8\textwidth]{plotted_derivative_funtion.png}
\end{figure}

\begin{figure}[H]
\caption{Section 6}
\centering
\includegraphics[width=.8\textwidth]{time_reversal.png}
\end{figure}

\begin{figure}[H]
\caption{Section 6}
\centering
\includegraphics[width=.8\textwidth]{time_shift_f(t-4).png}
\end{figure}

\begin{figure}[H]
\caption{Section 6}
\centering
\includegraphics[width=.8\textwidth]{time_shift_f(-t-4).png}
\end{figure}


\begin{figure}[H]
\caption{Section 6}
\centering
\includegraphics[width=.8\textwidth]{time_scale_f(t_div_2).png}
\end{figure}

\begin{figure}[H]
\caption{Section 6}
\centering
\includegraphics[width=.8\textwidth]{time_scale_f(2_mult_t).png}
\end{figure}


%Questions
\section*{Questions}

\begin{enumerate}
    \item Are the plots from  Part 3 Task 4 and Part 3 Task 5 identical? Is it possible for them to match? Explain why or why not?
    
    \begin{enumerate}
        \item They are some what identical. It is possible to make them look alike if the computer instructions match what a human can do.
    \end{enumerate}
    
    \item How does the correlation between the two plots (from Part 3 Task 4 and Part 3 Task 5)
change if you were to change the step size within the time variable in Task 5? Explain why
this happens.

    \begin{enumerate}
        \item If the step size increases the points that are graphed are farther away. That results in a graph that looks funky and not accurate. Likewise the smaller the step size is the more points that are graphed. Thus a cleaner looking graph.
    \end{enumerate}
    
    \item Leave any feedback on the clarity of lab tasks, expectations, and deliverables.
    
    \begin{enumerate}
        \item N/A
    \end{enumerate}
    
\end{enumerate}

%Conclusion
\section*{Conclusion}
In summary Python is a great and easy language to pick up. User defined functions are similar like C functions. The useful libraries for this lab were "numpy" and "matplotlib".



\end{document}