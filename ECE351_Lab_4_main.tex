%%%%%%%%%%%%%%%%%%%%%%%%%%%%%
% Carlos Santos             %    
% ECE 351-51                %
% Lab 4                     %
% 2/25/2020                 %
%                           %
%%%%%%%%%%%%%%%%%%%%%%%%%%%%%

\documentclass[12pt]{article}

% Language and font encoding
\usepackage[english]{babel}
\usepackage[utf8x]{inputenc}
\usepackage[T1]{fontenc}
\usepackage{graphicx}
\usepackage{amsmath}
\usepackage{caption}
\usepackage{float}
\usepackage{caption}
\usepackage{subcaption}
\usepackage{rotating}
\usepackage{setspace}


% Sets page size and margins
\usepackage[a4paper,top=3cm,bottom=2cm,left=3cm,right=3cm,marginparwidth=1.75cm]{geometry}

% Useful packages
\usepackage[colorinlistoftodos]{todonotes}
\usepackage[colorlinks=true, allcolors=blue]{hyperref}
\usepackage{listings}
\usepackage{gensymb}
\usepackage{graphicx} %package to manage images
\usepackage{listings}
\usepackage{color}

\definecolor{dkgreen}{rgb}{0,0.6,0}
\definecolor{gray}{rgb}{0.5,0.5,0.5}
\definecolor{mauve}{rgb}{0.58,0,0.82}

\lstset{frame=tb,
  language=Python,
  aboveskip=3mm,
  belowskip=3mm,
  showstringspaces=false,
  columns=flexible,
  basicstyle={\small\ttfamily},
  numbers=none,
  numberstyle=\tiny\color{gray},
  keywordstyle=\color{blue},
  commentstyle=\color{dkgreen},
  stringstyle=\color{mauve},
  breaklines=true,
  breakatwhitespace=true,
  tabsize=3
}

%Line Spacing
\setstretch{1.5}


%Info for Title Page
\title{ECE 351 Lab 4 Report \\ Section 52}
\date{February 25, 2020}
\author{Carlos Santos}
\begin{document}

%Make a Title Page
\vspace{\fill}
\maketitle
\vspace{\fill}
\clearpage

%Introduction
\section*{Introduction}

The purpose of this lab is to become more familiar using convolution to calculate a systems impulse response.

%Equations
\section*{Equations}

User Defined Functions: Where \[ f_{0} = 0 \]
\begin{equation}
    h_1(t) = e^{2t}*u(1-t)
\end{equation}{}
\begin{equation}
    h_2(t) = u(t-2) - u(t-6)
\end{equation}
\begin{equation}
    h_3(t) = cos(w_0t)*u(t)
\end{equation}

Hand Calculations:

\begin{equation}
    h_1(t)**u(t) = 0.5e^{2t}*u(1-t) + 0.5e^{2}*u(t-1) 
\end{equation}
\begin{equation}
    h_2(t)**u(t) = (r(t-2) - r(t-6))*u(t)
\end{equation}
\begin{equation}
    h_3(t)**u(t) = (1/w_0) * sin(w_0*t)*u(t)
\end{equation}


%Methodology
\section*{Methodology}

\begin{enumerate}
    \item First I imported code from Lab 3. I imported in the ramp,step, and convolution functions.
    \item I did all the required plots.
    \item Next, I did the convolutions by hand. 
    \item Then plot and compare the hand calculations with the computers results.
    \item Tweak the plots to show the functions properly.
\end{enumerate}

%Results
\section*{Results}

\begin{figure}[H]
\caption{User Defined Functions}
\centering
\includegraphics[width=.8\textwidth]{User_Defined_Functions.png}
\end{figure}

\begin{figure}[H]
\caption{Check with sig.convo}
\centering
\includegraphics[width=.8\textwidth]{check_with_sig_convolve.png}
\end{figure}

\begin{figure}[H]
\caption{Hand Calc Convolution}
\centering
\includegraphics[width=.8\textwidth]{hand_calc.png}
\end{figure}

%Questions
\section*{Questions}

\begin{enumerate}
    \item 1. Leave any feedback on the clarity of lab tasks, expectations, and deliverables.
    \begin{enumerate}
        \item Not much to say. This lab was pretty straight forward.
    \end{enumerate}
\end{enumerate}

%Conclusion
\section*{Conclusion}

In general this wasn't a bad lab. Some take aways are I am still not good at convolutions by hand. I should spend some more time on them. Also, when plugging in the hand calculation equations it's normal for them not to go down back to 0.

\end{document}